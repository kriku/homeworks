\documentclass[a4paper,11pt]{article}
\usepackage{indentfirst}
\usepackage{listings}
\usepackage{color}

\definecolor{mygreen}{rgb}{0,0.6,0}
\definecolor{mygray}{rgb}{0.5,0.5,0.5}
\definecolor{mymauve}{rgb}{0.28,0,0.82}
\definecolor{myblue}{rgb}{0.8,0.2,0.8}

\lstset{
  belowskip=-12pt, %\baselineskip
  backgroundcolor=\color{white},
  basicstyle=\ttfamily\footnotesize,
  breakatwhitespace=false,
  breaklines=true,
  captionpos=b,
  commentstyle=\color{mygreen},
  deletekeywords={...},
  escapeinside={\%*}{*)},
  extendedchars=true,
  % frame=single,
  keepspaces=true,
  keywordstyle=\color{myblue},
  language=Java,
  otherkeywords={*,...},
  % numbers=left,
  % numbersep=5pt,
  % numberstyle=\ttfamily\color{mygray},
  rulecolor=\color{black},
  showspaces=false,
  showstringspaces=false,
  showtabs=false,
  stepnumber=1,
  stringstyle=\color{mymauve},
  tabsize=2,
  title=\lstname
}

\author{Krikun Georgy}
\title{Software Architecture\\Homework 1}
\begin{document}
\maketitle
\newpage
\section{Question}
\textbf{a - Correct}
\begin{lstlisting}
public abstract class Square implements Shape {
	public abstract void setColor(String s);
}
\end{lstlisting}
In this case defined public abstract class with public abstract method setColor(String s)
This method defined without body, and don't violate syntax.\\

\textbf{b - Abstract methods do not specify a body}
\begin{lstlisting}
public abstract class Square implements Shape {
	public abstract void setColor(String s) {}
}
\end{lstlisting}
This variant trying define abstract method with empty body.
But for abstract methods we can't define body, ever.\\

\textbf{c - The type Shape cannot be the superclass of Square; a superclass must be a class}

\begin{lstlisting}
public abstract class Square extends Shape {
	public void setColor(String s) {}
}
\end{lstlisting}
For inheritance in Java we can extends some classes, but not interfaces.
We can implement interfaces or extend interfaces by another, but can't extend class with interface.\\

\textbf{d - Correct}

\begin{lstlisting}
public abstract class Square implements Shape {
	public void setColor(Integer i) {}
}
\end{lstlisting}
In this implementation we have correct definition of class Square, which implements interface Shape.\\

\textbf{e - Correct}

\begin{lstlisting}
public abstract class Square implements Shape {
	public void setColor(String s) {}
	public void setColor(Integer i) {}
}
\end{lstlisting}
For overloading methods in Java we can define different methods with same name, for different types and arity of parameters\\

\section{Question}

\textbf{a - correct}
\begin{lstlisting}
public Flower getType() { return this; }
\end{lstlisting}
Redefine method from Plant class, which return instance of Plant type but with new one, which return instance of Flower\\

\textbf{b - incorrect}
\begin{lstlisting}
public String getType() { return "this"; }
\end{lstlisting}
We can't override method, what returns incompatible with superclass type\\

\textbf{c - correct}
\begin{lstlisting}
public Plant getType() { return this;}
\end{lstlisting}
This method return Plant instance from Flower, downcast instance to Plant\\

\textbf{d - correct}
\begin{lstlisting}
public Tulip getType() { return new Tulip(); }
\end{lstlisting}
As soon as Tulip inherited from Plant\\

\textbf{e - incorrect}
\begin{lstlisting}
public String getType() { return this; }
\end{lstlisting}
This variant shows two violations:
\begin{itemize}
  \item
First one - type String not compatible with Plant type
  \item
Second one - \texttt{this} - instance of Flower type, but method declared with String return type
\end{itemize}

\section{Question}
\textbf{returns 9}\\
On first step created static field \texttt{y}, of \texttt{Uber} class, and initialize with value 2.
Then try create \texttt{Minor} instance, by calling constructor \texttt{Uber(int x)}.
This constructor call default constructor for \texttt{Uber()}, which multiply value \texttt{y} twice.
After this evaluation returns to \texttt{Minor()} constructor, which increment value \texttt{y} by three.
So for static field y we have value 9, and as soon as value public we can access it from other classes.
\newpage

\section{Question}
\textbf{a - incorrect}
\begin{lstlisting}
PitBull p2 = (PitBull) dog1;
\end{lstlisting}
\texttt{dog1} created by effective class \texttt{Dog}, we can't cast parent class to child,
because child class can be ``bigger'' than parent.\\

\textbf{b - correct}
\begin{lstlisting}
PitBull p2 = (PitBull) dog2;
\end{lstlisting}
In this case \texttt{dog2} instance created by \texttt{PitBull} constructor, so in memory we have
full \texttt{PitBull} instance, which downcast to \texttt{Dog} in under the hod, when we assign to \texttt{dog2}.
So when we explicitly cast instance to \texttt{PitBull}, we can do it, because we have \texttt{PitBull} instance in memory.\\

\textbf{c - incorrect}
\begin{lstlisting}
PitBull p2 = dog2;
\end{lstlisting}
But in this case, we didn't cast it explicitly. Compiler don't know, how to cast that, because in general
we can't say, that instance on this reference is \texttt{PitBull}.\\

\section{Question}

\textbf{a - incorrect}
\begin{lstlisting}
x2.do2();
\end{lstlisting}
The method \texttt{do2()} is undefined for the type \texttt{X}
\texttt{x2} is instance of \texttt{X} class, so when we call \texttt{Y} constructor,
we create \texttt{Y} instance, but cast it to \texttt{X} class (we can do it because \texttt{Y} extends \texttt{X})
But \texttt{X} class haven't got \texttt{do2} method.\\

\textbf{b - correct}
\begin{lstlisting}
((Y)x2).do2();
\end{lstlisting}
This is legal call of \texttt{do2} method, \texttt{Y} class, because we create \texttt{x2} object with \texttt{Y} constructor.\\

\textbf{c - incorrect}
\begin{lstlisting}
(Y)x2.do2();
\end{lstlisting}
Cast priority less than methods call.
So we can't call method before cast to \texttt{Y} type.
The method \texttt{do2()} is undefined for the type \texttt{X}.

\end{document}
